\documentclass{article}
\usepackage[T1]{fontenc}
\usepackage{lmodern}
\usepackage{siunitx}
\usepackage{amssymb}
\usepackage{enumitem}
\usepackage{titlesec}
\usepackage{parskip}
\usepackage{framed}
\usepackage{color}
\usepackage[colorlinks]{hyperref}

\definecolor{shadecolor}{rgb}{0.757, 0.882, 0.925}
\definecolor{shadecolor}{rgb}{0.757, 0.882, 0.925}

\renewcommand\thesection{}

\renewcommand\thesubsection{\arabic{subsection}}

\titleformat{\subsection}{\normalfont\bfseries}{Q\thesubsection}{1em}{}


\begin{document}

Dear Dr. Anderson,

The interactive review of your manuscript "Depth dependent three-layer model for
the surface second-harmonic generation yield" submitted to Frontiers in
Materials, section Thin Solid Films has now been activated.

The reviewers recommended that you make substantial amendments to your
manuscript. Please respond within the next 35 days to all comments raised by the
reviewers in the online review forum. If necessary, you can also submit a
revised version of your manuscript at that time. There can be more than one
iteration between authors and reviewers, but only when all comments by each
reviewer have been addressed successfully can the review be finalized.

To access the review forum and respond to the reviewers, please click \href{http
://www.frontiersin.org/Review/EnterReviewForum.aspx?activationno=756b11ac-a551-4
580-8448-3afb497009d2}{here}.

\begin{description}[itemsep=-4pt]
\item[Manuscript title:] Depth dependent three-layer model for the surface
second-harmonic generation yield
\item[Manuscript ID:] 244688 
\item[Authors:] Sean Martin Anderson, Bernardo Mendoza 
\item[Journal:] Frontiers in Materials, section Thin Solid Films 
\item[Article type:] Original Research 
\item[Submitted on:] 28 Nov 2016 
\item[Interactive review started on:] 20 Feb 2017 
\end{description}

Please do not hesitate to contact us if you have any questions. Your timely
response would be much appreciated. Please note that if we do not hear from you
by the revision deadline, your manuscript may be treated as a new submission, as
we cannot hold manuscripts in review without any updates from the authors.

With best regards, 

Alina Vladescu\\
Guest Associate Editor,\\
\url{www.frontiersin.org}

%%%%%%%%%%%%%%%%%%%%%%%%%%%%%%%%%%%%%%%%%%%%%%%%%%%%%%%%%%%%%%%%%%%%%%%%%%%%%%%%

\section{Reviewer 1}

%%%%%%%%%%%%%%%%%%%%%%%%%%%%%%%%%%%%%%%%%%%%%%%%%%%%%%%%%%%%%%%%%%%%%%%%%%%%%%%%

\subsection{Does this manuscript conform to the definition below of Original
Research articles? If not, please contact the Frontiers Editorial Office
(editorial.office@frontiersin.org).}

Yes, it has the right form, but it is very close to the work of the same authors
published in PRB (2016).

\begin{shaded}
We have revised the manuscript based on the timely suggestions provided below. We slimmed down the Introduction and Theory sections, added significant discussion to the Results, enhanced Fig. 2, and added a new Fig. 3. We consider that these revisions will help distinguish this work from our previous PRB paper. 
\end{shaded}

%%%%%%%%%%%%%%%%%%%%%%%%%%%%%%%%%%%%%%%%%%%%%%%%%%%%%%%%%%%%%%%%%%%%%%%%%%%%%%%%

\subsection{Is the language, specifically the grammar, of sufficient quality?
If no, please specify if the authors should send this manuscript to an expert in
English editing and academic writing.}

Yes, almost entirely free from mechanical issues.

\begin{shaded}
N/A
\end{shaded}

%%%%%%%%%%%%%%%%%%%%%%%%%%%%%%%%%%%%%%%%%%%%%%%%%%%%%%%%%%%%%%%%%%%%%%%%%%%%%%%%

\subsection{What are the main findings reported in this manuscript?}

The authors describe a (fairly modest) extension of a model
they published recently in Physical Review B (``Three-layer model for the
surface second-harmonic generation yield including multiple reflections,''
\emph{Phys.\ Rev.\ B} \textbf{94} (2016) 115314). The extension appears to
be the treatment of the surface layer $\ell$ as a set of layers each
generating second-harmonic polarization with its own susceptibility tensor
derived from \emph{ab initio} calculations. Each source polarization
generates a reflected wave at $2\omega$ that is computed by including an
infinite series of reflections from the interface between $\ell$ and the
vacuum (v), on one hand, and the interface between $\ell$ and the bulk (b),
on the other. This approach differs from that in the prior work, which
modeled the nonlinear polarization as being uniform throughout $\ell$.

\begin{shaded}
N/A
\end{shaded}

%%%%%%%%%%%%%%%%%%%%%%%%%%%%%%%%%%%%%%%%%%%%%%%%%%%%%%%%%%%%%%%%%%%%%%%%%%%%%%%%

\subsection{Does the title clearly and precisely reflect the findings of the
manuscript, as described in the author guidelines?}

Sure

\begin{shaded}
N/A
\end{shaded}

%%%%%%%%%%%%%%%%%%%%%%%%%%%%%%%%%%%%%%%%%%%%%%%%%%%%%%%%%%%%%%%%%%%%%%%%%%%%%%%%

\subsection{Please comment on the Abstract section. Potential aspects to
consider: appropriateness of context, purpose of study.}

The abstract should report the significance of the new work; it merely concludes
that the work was done.

\begin{shaded}
We have revised the abstract to better represent the significance of this work.
\end{shaded}

%%%%%%%%%%%%%%%%%%%%%%%%%%%%%%%%%%%%%%%%%%%%%%%%%%%%%%%%%%%%%%%%%%%%%%%%%%%%%%%%

\subsection{Please comment on the Introduction section. Potential aspects to
consider: appropriateness of context, purpose of study.}

Fine, although it could be condensed by reference to the authors' PRB (2016),
thus freeing up space to focus on what is *new* about this work.

\begin{shaded}
We have removed a paragraph from the Introduction, and added content to the Results that expand on the significance of this article over our previous work.
\end{shaded}

%%%%%%%%%%%%%%%%%%%%%%%%%%%%%%%%%%%%%%%%%%%%%%%%%%%%%%%%%%%%%%%%%%%%%%%%%%%%%%%%

\subsection{Please comment on the Material and Methods section. Potential
aspects to consider: objective errors, correct choice of methods, comprehensive
description of methods, accuracy of procedures, quality of figures and tables}

\begin{enumerate}[label=\alph*.]
\item A primary feature of the three-layer model is a reflection from the
$\ell$b buried interface at the second-harmonic frequency, giving rise to
important depth-dependent contributions to the radiated field. As best I
can understand, the dielectric function used to describe $\ell$ is an
average of the computed \emph{ab initio} linear dielectric properties,
which presumably shift smoothly from their bulk values to surface values
as depth $z$ approaches zero (the vacuum interface). The importance of
reflections from the $\ell$b interface is dubious for two reasons. First,
the dielectric contrast between an intermediate dielectric function for
the layer and the bulk is likely to be very small, leading to negligible
reflection from this (computational) interface. Second, the \emph{ab
initio} computation approaches bulk centrosymmetry by a depth of
\SI{3.6}{nm}, by which we can infer that the dielectric function differs
there negligibly from the bulk value. Insofar as reflections from a stack
of layers with gradually varying dielectric properties is even smaller
than from an abrupt change in index, I would appreciate a more thorough 
discussion of linear dielectric properties assumed in the modeling.

\begin{shaded}
We have added a short paragraph in the Results that discuss these considerations for $\epsilon_{\ell}$ and $\epsilon_{b}$, between lines 226-233 of the manuscript. While the dielectric functions become similar as we go deeper into the slab, there are still differences between the two that provide enough contrast to warrant reflections from the $\ell b$ interface.

Bear in mind that with the three layer model, we can choose a bulk region that is completely different than the thin layer (i.e. a thin film over a substrate). In that case, the dielectric functions may have very marked differences between the regions. We have added a paragraph at the end of the Results that adds some remarks concerning this point.
\end{shaded}

\item The primary difference between this work and that published in the PRB
cited above would appear to be allowing $\chi(2\omega)$ to vary with $z$
in the layer $\ell$. In other respects, it appears to be identical to the
PRB work. In view of this tight relation, I find it odd that the authors
do not compare this work to the previous work in Fig.\ 3.

\begin{shaded}
This figure is now Fig. 4 in the revised manuscript. We agree and have added the results featured in our PRB paper as ``Average ($d = 10$ nm).'' Both curves labeled ``Average '' are indeed calculated using the procedure detailed in our previous work, and are neither depth dependent ($\chi^{\mathrm{abc}}$ is for the complete half-slab) nor \emph{ab initio} (the layer depth $d$ is a free parameter). We have added clarifying remarks to this effect in the Results section, and in the figure caption.
\end{shaded}

\item Figure 2 shows an interesting comparison of the nonlinear
susceptibility as a function of $2\hslash \omega$ for the hydrogen layer,
the topmost pair of silicon layers ($z_{2}+z_{3}$) and the deepest pair of
silicon layers considered ($z_{24}+z_{25}$). The deepest layer pair
generates a surprisingly large signal in comparison to the topmost pair,
and it has a rather different spectrum. It would perhaps be appropriate to
include some intermediate curves and to tease out the physical origin of
these spectra.

\begin{shaded}
We have redone Fig. 2 to better explain this point. Specifically, we have added insets that depict the individual contributions from the Si layers at $z_{2}$, $z_{3}$, $z_{24}$, and $z_{25}$. These insets clearly present how the the first two Si layers produce very similar spectra that constructively interfere, thus producing an overall signal enhancement. The last two Si layers produce spectra that are essentially opposite in sign, and mostly cancel out with a relatively small contribution to the overall SH signal.

We have also added a new figure, Fig. 3, that relates directly to Eq. (12). Essentially, the contribution from the last two layers to the overall spectrum is quite small. This discussion is included between lines 160-163.
\end{shaded}

\item The bottom line is that the authors would do the present work and the
community justice by more carefully stressing the important difference
between this and their previous work, and spending comparatively more
space in the paper teasing out the significance of the present
work. Indeed, most of the theoretical section could be treated with
reference to the PRB paper, freeing up space for the comparison.

\begin{shaded}
Thank you for these comments and suggestions.

As mentioned above, we have reduced the Introduction and Theory sections by making extensive reference to our PRB publication. We have also expanded the Results considerably in order to highlight the significance of the present work. Overall, we think that these modifications do help this manuscript stand out from, and build upon, our previous PRB paper. 
\end{shaded}

\end{enumerate}


%%%%%%%%%%%%%%%%%%%%%%%%%%%%%%%%%%%%%%%%%%%%%%%%%%%%%%%%%%%%%%%%%%%%%%%%%%%%%%%%

\subsection{Are the statistical methods used valid?}

Not Applicable

\begin{shaded}
N/A
\end{shaded}

%%%%%%%%%%%%%%%%%%%%%%%%%%%%%%%%%%%%%%%%%%%%%%%%%%%%%%%%%%%%%%%%%%%%%%%%%%%%%%%%

\subsection{Has the work been conducted in conformity with the ethical
standards of the field?}

Yes

\begin{shaded}
N/A
\end{shaded}

%%%%%%%%%%%%%%%%%%%%%%%%%%%%%%%%%%%%%%%%%%%%%%%%%%%%%%%%%%%%%%%%%%%%%%%%%%%%%%%%

\subsection{For research involving human subjects or animals, do the
author(s) identify the committee approving the studies and provide confirmation
that all studies conform to the relevant regulatory standards?}

Not Applicable

\begin{shaded}
N/A
\end{shaded}

%%%%%%%%%%%%%%%%%%%%%%%%%%%%%%%%%%%%%%%%%%%%%%%%%%%%%%%%%%%%%%%%%%%%%%%%%%%%%%%%

\subsection{For research involving biohazards, have the correct standard
procedures been carried out?}

Not Applicable

\begin{shaded}
N/A
\end{shaded}

%%%%%%%%%%%%%%%%%%%%%%%%%%%%%%%%%%%%%%%%%%%%%%%%%%%%%%%%%%%%%%%%%%%%%%%%%%%%%%%%

\subsection{Please comment on the Results section. Potential aspects to
consider: objective errors, correct presentation of results, quality of figures
and tables}

The results section could be be a stronger focus of the paper.

\begin{shaded}
We have expanded the Results considerably in order to highlight the significance of the present work.
\end{shaded}

%%%%%%%%%%%%%%%%%%%%%%%%%%%%%%%%%%%%%%%%%%%%%%%%%%%%%%%%%%%%%%%%%%%%%%%%%%%%%%%%

\subsection{For any complementary data (e.g. nucleotide/amino acid sequences,
crystallographic or NMR data, microarray data) submitted to an online repository
or database, do the author(s) provide the accession number?}

Not Applicable

\begin{shaded}
N/A
\end{shaded}

%%%%%%%%%%%%%%%%%%%%%%%%%%%%%%%%%%%%%%%%%%%%%%%%%%%%%%%%%%%%%%%%%%%%%%%%%%%%%%%%

\subsection{Please comment on the Discussion section. Potential aspects to
consider: adequate discussion of research questions or hypothesis (posed in
introduction), conclusions supported by data, exhaustive discussion of
previously published material (in context to current study)}

Referee 1 proposes the same comments and suggestions for the Material and
Methods section (see above).

\begin{shaded}
We have addressed these issues in our response to the comments on the Material and Methods section, described above.
\end{shaded}

%%%%%%%%%%%%%%%%%%%%%%%%%%%%%%%%%%%%%%%%%%%%%%%%%%%%%%%%%%%%%%%%%%%%%%%%%%%%%%%%

\subsection{Please add here any further comments on this manuscript.}

Mechanics: it might be useful to report that the authors assume $\mu = 1$
throughout. Two small typos on p. 8: ``is an good'' $\rightarrow$ ``is a good'';
``results that is independent'' $\rightarrow$ ``results that are independent''

\begin{shaded}
We have added a sentence on line 79 clarifying that $\mu = 1$.

We have corrected these two typographic errors in the manuscript. We also corrected a typo in Fig. 3, changing $d = 3.5$ nm to $d = 3.6$ nm.

Thank you for the suggestions and comments. We have incorporated them into the manuscript as described above. All relevant changes to the manuscript have been marked in red for easy reference.
\end{shaded}

%%%%%%%%%%%%%%%%%%%%%%%%%%%%%%%%%%%%%%%%%%%%%%%%%%%%%%%%%%%%%%%%%%%%%%%%%%%%%%%%
%%%%%%%%%%%%%%%%%%%%%%%%%%%%%%%%%%%%%%%%%%%%%%%%%%%%%%%%%%%%%%%%%%%%%%%%%%%%%%%%
%%%%%%%%%%%%%%%%%%%%%%%%%%%%%%%%%%%%%%%%%%%%%%%%%%%%%%%%%%%%%%%%%%%%%%%%%%%%%%%%


\section{Reviewer 2}
\setcounter{subsection}{0}

%%%%%%%%%%%%%%%%%%%%%%%%%%%%%%%%%%%%%%%%%%%%%%%%%%%%%%%%%%%%%%%%%%%%%%%%%%%%%%%%

\subsection{Does this manuscript conform to the definition below of Original
Research articles? If not, please contact the Frontiers Editorial Office
(editorial.office@frontiersin.org).}

Yes. Please take into account that the Results and Discussion are merged in the
section entitled ``The SSHG yield of the Si(111)1$\times$1:H Surface for
\emph{p}-in, \emph{P}-out polarization''

\begin{shaded}
N/A
\end{shaded}

%%%%%%%%%%%%%%%%%%%%%%%%%%%%%%%%%%%%%%%%%%%%%%%%%%%%%%%%%%%%%%%%%%%%%%%%%%%%%%%%

\subsection{Is the language, specifically the grammar, of sufficient quality?
If no, please specify if the authors should send this manuscript to an expert in
English editing and academic writing.}

Yes

\begin{shaded}
N/A
\end{shaded}

%%%%%%%%%%%%%%%%%%%%%%%%%%%%%%%%%%%%%%%%%%%%%%%%%%%%%%%%%%%%%%%%%%%%%%%%%%%%%%%%

\subsection{What are the main findings reported in this manuscript?}

It is presented a refined formalism aimed to calculate SSHG spectra. The results
obtained through the developed theory were compared to experimental data
concerning surface-SHG on Si(111)(1x1):H.

\begin{shaded}
N/A
\end{shaded}

%%%%%%%%%%%%%%%%%%%%%%%%%%%%%%%%%%%%%%%%%%%%%%%%%%%%%%%%%%%%%%%%%%%%%%%%%%%%%%%%

\subsection{Does the title clearly and precisely reflect the findings of the
manuscript, as described in the author guidelines?}

Yes, the title is appropriate

\begin{shaded}
N/A
\end{shaded}

%%%%%%%%%%%%%%%%%%%%%%%%%%%%%%%%%%%%%%%%%%%%%%%%%%%%%%%%%%%%%%%%%%%%%%%%%%%%%%%%

\subsection{Please comment on the Abstract section. Potential aspects to
consider: appropriateness of context, purpose of study.}

No answer given.

\begin{shaded}
N/A
\end{shaded}

%%%%%%%%%%%%%%%%%%%%%%%%%%%%%%%%%%%%%%%%%%%%%%%%%%%%%%%%%%%%%%%%%%%%%%%%%%%%%%%%

\subsection{Please comment on the Introduction section. Potential aspects to
consider: appropriateness of context, purpose of study.}

The introduction missed some more recent references concerning Surface SHG on
thin films. I suggest to add the following:

\begin{itemize}
\item W. M. M. Kessels et al., ``Spectroscopic second harmonic generation
measured on plasma-deposited hydrogenated amorphous silicon thin films'', Appl.
Phys. Lett. 85, 4049 (2004)
\item I. M. P. Aarts et al., ``Probing hydrogenated amorphous silicon surface
states by spectroscopic and real-time second-harmonic generation'', Phys. Rev. B
73, 45327 (2006)
\item S. Lettieri et al., ``Second harmonic generation analysis in hydrogenated
amorphous silicon nitride thin films'', Appl. Phys. Lett. 90, 21919 (2007)
\item J.J.H Gielis etal., ``Optical second-harmonic generation in thin film 
systems'', J. of Vacuum Science and Technology A 26, 1519 (2008)
\end{itemize}

\begin{shaded}
The mentioned references are indeed both recent and relevant, so we have added them to the introduction on lines 43-47.
\end{shaded}

%%%%%%%%%%%%%%%%%%%%%%%%%%%%%%%%%%%%%%%%%%%%%%%%%%%%%%%%%%%%%%%%%%%%%%%%%%%%%%%%

\subsection{Please comment on the Material and Methods section. Potential
aspects to consider: objective errors, correct choice of methods, comprehensive
description of methods, accuracy of procedures, quality of figures and tables}

No answer given.

\begin{shaded}
N/A
\end{shaded}

%%%%%%%%%%%%%%%%%%%%%%%%%%%%%%%%%%%%%%%%%%%%%%%%%%%%%%%%%%%%%%%%%%%%%%%%%%%%%%%%

\subsection{Are the statistical methods used valid?}

Not Applicable

\begin{shaded}
N/A
\end{shaded}

%%%%%%%%%%%%%%%%%%%%%%%%%%%%%%%%%%%%%%%%%%%%%%%%%%%%%%%%%%%%%%%%%%%%%%%%%%%%%%%%

\subsection{Has the work been conducted in conformity with the ethical
standards of the field?}

Not Applicable

\begin{shaded}
N/A
\end{shaded}

%%%%%%%%%%%%%%%%%%%%%%%%%%%%%%%%%%%%%%%%%%%%%%%%%%%%%%%%%%%%%%%%%%%%%%%%%%%%%%%%

\subsection{For research involving human subjects or animals, do the
author(s) identify the committee approving the studies and provide confirmation
that all studies conform to the relevant regulatory standards?}

Not Applicable

\begin{shaded}
N/A
\end{shaded}

%%%%%%%%%%%%%%%%%%%%%%%%%%%%%%%%%%%%%%%%%%%%%%%%%%%%%%%%%%%%%%%%%%%%%%%%%%%%%%%%

\subsection{For research involving biohazards, have the correct standard
procedures been carried out?}

Not Applicable

\begin{shaded}
N/A
\end{shaded}

%%%%%%%%%%%%%%%%%%%%%%%%%%%%%%%%%%%%%%%%%%%%%%%%%%%%%%%%%%%%%%%%%%%%%%%%%%%%%%%%

\subsection{Please comment on the Results section. Potential aspects to
consider: objective errors, correct presentation of results, quality of figures
and tables}

No answer given.

\begin{shaded}
N/A
\end{shaded}

%%%%%%%%%%%%%%%%%%%%%%%%%%%%%%%%%%%%%%%%%%%%%%%%%%%%%%%%%%%%%%%%%%%%%%%%%%%%%%%%

\subsection{For any complementary data (e.g. nucleotide/amino acid sequences,
crystallographic or NMR data, microarray data) submitted to an online repository
or database, do the author(s) provide the accession number?}

Not Applicable

\begin{shaded}
N/A
\end{shaded}

%%%%%%%%%%%%%%%%%%%%%%%%%%%%%%%%%%%%%%%%%%%%%%%%%%%%%%%%%%%%%%%%%%%%%%%%%%%%%%%%

\subsection{Please comment on the Discussion section. Potential aspects to
consider: adequate discussion of research questions or hypothesis (posed in
introduction), conclusions supported by data, exhaustive discussion of
previously published material (in context to current study)}

line 98 -the authors stated ``\ldots The calculation of position dependence of
the linear field is a complicated problem worth pursuing, but is outside the
scope of this work\ldots'' The evidenced complexity of the linear EM field
along the depth should be at least briefly discussed

\begin{shaded}
We have added a brief remark with a relevant reference on the nature of this calculation, on lines 86-88 of the Theory section.
\end{shaded}

line 190- the authors evidenced that the E2 resonance in the theoretical SHG
spectrum of Si(111)(1$\times$1):H is markedly blueshifted with respect the
experimental feature. Such a discrepancy should be clarified.

\begin{shaded}
A paragraph has been added to the Results section (lines 191-209) that clarifies this discrepancy in considerable detail.
\end{shaded}

%%%%%%%%%%%%%%%%%%%%%%%%%%%%%%%%%%%%%%%%%%%%%%%%%%%%%%%%%%%%%%%%%%%%%%%%%%%%%%%%

\subsection{Please add here any further comments on this manuscript.}

The manuscript is worth to be published following the abovementioned remarks.

\begin{shaded}
Thank you for the suggestions and comments. We have incorporated them into the manuscript as described above. All relevant changes to the manuscript have been marked in red for easy reference. 
\end{shaded}

\end{document}
