\subsection{Length Gauge Formalism}
According to ~\onlinecite{ismailPRL01}, we first start with the interaction Hamiltonian expressed in the velocity gauge, containing the nonlocal parts
$V^{nl}$ and $S$. Within the dipole approximation and using a gauge transformation, it can be transformed into an effective Hamiltonian \cite{Veniard, unpublished}
\begin{equation}
\tilde H_{I}(t)=-e\mathbf{r}\cdot \mathbf{E}(t).  \label{rde}
\end{equation}
The treatment of the position operator $\bfr$ for extended Bloch states is problematic and has been discussed in Refs. \cite{adamsJCP53,blountSSP62} 
{\color{\chon}. Following Ref.~\onlinecite{sipePRB00}, we take} the matrix elements of Eq.~\eqref{rhop} with the Hamiltonian $\tilde H_{I}(t)$ of
Eq.~\eqref{rde}, we obtain 
$(\tilde{\rho}^{(1)}(t))_{nm}=B_{nm}^{b}E^{b}(\omega)e^{i(\omega^\sigma_{nm}-\tilde\omega)t}$,
with 
\begin{equation}\label{j.1}
B_{nm}^{b}=\frac{e}{\hbar }\frac{f_{mn}r_{nm}^{b}}{\omega^\Sigma_{nm}-\tilde\omega},
\end{equation}
and 
\begin{eqnarray}\label{j.2}
(\tilde{\rho}^{(2)}(t))_{nm} &=&\frac{e}{i\hbar }\frac{1}{\omega^\Sigma_{nm}-2\tilde\omega}\bigg[
i\sum_{q }\Big(r_{nq }^{b}B_{q m}^{c}-B_{nq}^{c}r_{q m}^{b}
\Big)  
-(B_{nm}^{c})_{;k^{b}}\bigg]E^{b}(\omega)E^{c}(\omega)e^{i(\omega^\Sigma_{nm}-2\tilde\omega)t}
.
\end{eqnarray}
$\bfr$ is split into the {\it intraband} 
($\bfr_i$) and {\it interband} ($\bfr_e$) parts, where 
$\bfr=\bfr_i+\bfr_e$. 
For $\bfr_e$ one uses
\begin{align}\label{rnminn}
\bra{n\bfk}\bfr_e\ket{m\bfk'} &=
(1-\gd_{nm})\gd(\bfk-\bfk')\bfgxi_{nm}(\bfk) 
,
\end{align}
such that $\bfr_{e,nm}=0$ for $n=m${\color{\chon} .
>From Eq.~\eqref{mv}} with $H_0\to
H^\Sigma_0$, we obtain
\begin{align}\label{pmnrmn}
\bfr_{e,nm}(\bfk) =
\bfgxi_{nm}(\bfk)\equiv 
\bfr_{nm}(\bfk) 
&=
\frac{\bfv^\gs_{nm}(\bfk)}{i\go^\gs_{nm}(\bfk)}
\quad\quad n\notin D_m 
,
\end{align}  
where we defined
$\go^\gs_{nm}(\bfk)\equiv\go^\gs_n(\bfk)-\go^\gs_m(\bfk)$, and
$D_m$ are all the possible degenerate $m$-states. 
%We recognize that for each instance in which 
%$n\notin D_m$ or $n\ne m$, $\bfr_{nm}(\bfk)$ are interband matrix
%elements.
For the intraband part, {\color{\chon} $\bfr_i$ only appears in
commutators during the derivation of
the optical response. We use},\cite{aversaPRB95}
\begin{equation}\label{conmri3n}
\bra{n\bfk}[\bfr_i,\calo]\ket{m\bfk'}
=i\gd(\bfk-\bfk')(\calo_{nm})_{;\bfk}
,
\end{equation}  
where
\begin{equation}\label{gendevnn}
(\calo_{nm})_{;\bfk}=
\nabla_\bfk 
\calo_{n m}(\bfk) 
- 
i 
\calo_{nm}(\bfk) 
\left(
\bfgxi_{nn}(\bfk) 
-
\bfgxi_{mm}(\bfk) 
\right) 
,
\end{equation} 
is
the generalized derivative 
of the operator $\calo$. 
The vectors $\bfgxi_{nn}(\bfk)$ are defined in 
Ref.~\onlinecite{aversaPRB95} {\color{\chon} though} they do not need to be 
calculated explicitly in what follows. 

Before {\color{\chon} continuing,} 
we derive a key result for the length gauge formulation. 
{\color{\chon} Again, using} $H^\Sigma_0$ in
Eq.~\eqref{mv} we obtain
\begin{align}\label{vop2}
\bfv^\gs&=
\bfv 
+
\bfv^\nl 
+\bfv^S
=
\bfv^\lda 
+\bfv^S 
,
\end{align}
where we have defined 
\begin{subequations}
\begin{align}\label{ve}
\bfv 
&=\frac{\bfp}{m_e},
\\\label{vnl}
\bfv^\nl 
&=
\frac{1}{i\hbar}[\bfr,V^\nl],
\\\label{vs}
\bfv^S
&=
\frac{1}{i\hbar}[\bfr, S],
\\\label{vlda}
\bfv^\lda 
&=
\bfv 
+\bfv^\nl 
,
\end{align}  
\end{subequations}
 with $[r^a,p^b]=i\hbar\gd_{ab}$, where $\gd_{ab}$ is the Kronecker delta.
Using Eq.~\eqref{hats} we obtain 
\begin{equation}\label{chon.2} 
\bfv^S_{nm}=i\Delta f_{mn}\bfr_{nm},
\end{equation}
with $f_{nm}\equiv f_n-f_m$,
where we see that $\bfv^S_{nn}=0$. From Eq.~\eqref{pmnrmn} and Eq.~\eqref{vop2} {\color{\chon} it} follows that
\begin{align}\label{chon.10}
\bfr_{nm}(\bfk) 
=
\frac{\bfv^\gs_{nm}(\bfk)}{i\go^\gs_{nm}(\bfk)}
=
\frac{\bfv^\lda_{nm}(\bfk)}{i\go^\lda_{nm}(\bfk)}
\quad\quad n\notin D_m 
. 
\end{align}
The matrix elements 
of $\bfr_{nm}(\bfk)$ are {\color{\chon} identical using either} 
the LDA or {\color{\chon} scissored} 
Hamiltonian, {\color{\chon} thus negating the} need to label them.
Of course, {\color{\chon} it} is more convenient to calculate them
through $\bfv^\lda_{nm}(\bfk)$ {\color{\chon} which 
includes} only the contribution of 
$\bfv^\nl_{nm}(\bfk)$. These {\color{\chon} can} be readily
calculated
for 
fully separable nonlocal pseudopotentials in the 
Kleinman-Bylander 
form,\cite{francesco,mottaCMS10,kleinmanPRL82,adolphPRB96}{\color{\chon}.}
{\color{\chon} The} advantage of using the electron density operator along with the length gauge formalism for 
{\color{\chon} calculating linear and nonlinear} optical responses, for the scissored Hamiltonian, 
resides in the ease with {\color{\chon} which the scissors operator
can be introduced into the calculation by simply using the unscissored LDA Hamiltonian, $H_0^{\mathrm{LDA}}$,
for the unperturbed system 
with $-e\mathbf{r}\cdot \mathbf{E}(t)$ as the interaction{\color{\chon}. We stress that within the length gauge, 
we need only replace $\omega^{\mathrm{LDA}}_{n}$ with 
$\omega_{n}^{\Sigma}$ at the end of the derivation}
to obtain the scissored results for any 
susceptibility expression, whether linear or nonlinear \cite{PRBNastos}. 


We have used the fact that for a cold semiconductor $\partial
f_{n}/\partial \mathbf{k}=0$ and thus the intraband contribution to the linear
term vanishes identically. 
%In Appendix \ref{gender} we show how the generalized derivative of 
%$(B^a_{nm})_{;k^b}$ can be obtained.
{\color{\chon} Note that the indices in Eq.~\eqref{j.2} are all} different from each
other. {\color{\chon} This is due to the $f_{nm}$ factor in Eq.~\eqref{j.1}, 
and therefore $B^a_{nn}=0$}. The dependence {\color{\chon} on} $\bfk$ 
of all quantities is implicitly understood from 
{\color{\chon} this point forward.}