From prb@aps.org Wed Dec 17 09:19:00 2014
Date: Tue, 16 Dec 2014 21:23:48 -0500
From: prb@aps.org
To: bms@cio.mx
Subject: {Spam?} Your_manuscript BZ12227 Anderson

    [ The following text is in the "UTF-8" character set. ]
    [ Your display is set for the "ISO-8859-1" character set.  ]
    [ Some characters may be displayed incorrectly. ]

Re: BZ12227
    Theory of surface second-harmonic generation for  semiconductors
    including effects of nonlocal operators
    by Sean M. Anderson, Nicolas Tancogne-Dejean, Bernardo S. Mendoza, et
    al.

Dear Dr. Mendoza,

The above manuscript has been reviewed by two of our referees. Comments
from the reports appear below.

These comments suggest that considerable revision of your manuscript is
in order. When you resubmit your manuscript, please include a summary
of any changes made and a succinct response to all recommendations or
criticisms contained in the reports.

Yours sincerely,

Athanasios Chantis
Assistant Editor
Physical Review B
Email: prb@aps.org
http://journals.aps.org/prb/

PRB Rapid Communications: Quality and Speed
http://journals.aps.org/prb/rapids

All PRB Editors' Suggestions feature prominently at
http://journals.aps.org/prb/

----------------------------------------------------------------------
Report of the First Referee -- BZ12227/Anderson
----------------------------------------------------------------------

The authors present a new scheme to calculate surface second-harmonic 
generation including the effects of non-local operators like the 
non-local part of the pseudo-potentials and the scissor-operator. 
Their approach reduces to the bulk one when the cut-function goes to 
one. They validated their new formulation on a clear example the 
Si(001)2x1 surface. I think their approach is correct and worth to be 
published in Physical Review B, I have only few comments and remarks. 

Comments are remarks: 

1) The authors claim that in this new formulation the scissor operator 
is correctly implemented respect to previous attempts to include it in 
the surface SHG response. Since some of the authors of the present 
paper are also author of the previous approaches (PRB 63, 2054406; PRB 
74, 075318) that “... incorrectly implemented” scissor operator, may 
they explain more in details what was wrong in the old papers and why 
now it is correct. 

2) The idea behind the scissor operator is to shift all the conduction 
bands of the same amount. While I can believe that this approximation 
is justified in a bulk material, why it should be in case for a 
surface? The shape and localization of surface states is different 
from the one of the bulk. There are cases where this approximation has 
been tested or compared with a full GW calculation for a surface? 

3) At the top of page 6 the authors write that X2 satisfies intrinsic 
permutation symmetry and then they put an equation with the same X2 in 
both sides. They should correct this mistake.

----------------------------------------------------------------------
Report of the Second Referee -- BZ12227/Anderson
----------------------------------------------------------------------

The authors present a theory of the second-harmonic generation at 
surfaces. The underlying electronic structures are calculated within 
the density functional theory. They discuss 

(i) the effect of quasiparticle corrections in the framework of the 
scissors-operator approach, 

(ii) the correct description of the optical-transition operator, and 

(iii) the reduction to the surface response by a cut function. 

The approach is timely and reasonable. The discussion of the three 
effects (i), (ii), and (iii) in combination is new. An application of 
the developed SHG approach to the Si(111)2x1 surface is demonstrated 
as a model case. Therefore, I claim that the publication of the 
manuscript in Physical Review B as a regular article can be suggested. 

However, prior to publication the article needs a substantial 
revision. It concerns the presentation of the theory, the removal of 
some misleading comments, and the discussion of the three effects. 
More in detail: 

- In one of their own papers by one of the authors, C. Castillo et 
al., PRB 68, 041310(R) (2003), the importance of the elimination of 
the bottom layers of the slab in the case of linear optics has been 
pointed out. Can this argument simply be applied to non-linear optics, 
where at least three light beams have to be discussed? 

- The reduction of the spectral strength in Fig. 4 in the two lower 
panels is a consequence of the inclusion of the scissors operator in 
the optical matrix elements. Am I right? Is this effect really real? 
Other treatments, e.g. in Ref. [36] and [54], claim that the scissors 
operator should not be taken into the reformulation of the optical 
matrix elements. 

- Abstract / Introduction: cancel “for the first time” 

- Abstract: cancel: “Our scheme … is exact” 

- Formulation in introduction “Radiation can occur …” is wrong 

- Sec.II.B: The description of light-matter interaction by the 
Hamiltonian (6) is invalid, because Bloch matrix elements of the space 
operator cannot be computed. However, another formulation as given in 
M. Gajdos et al., PRB 73, 045112 (2006) is possible. This 
representation is more convenient because no non-local contributions 
to the optical matrix element occur. A comparison as in B. Adolph et 
al., PRB 63, 125108 (2001), done for linear optics could also be 
helpful in the case of SHG. 

- The k-integration in 3D as indicated in many formulas from (18) and 
below does not fit to the slab approximation. Usually, only a 2D 
integration in k-space has to be performed.