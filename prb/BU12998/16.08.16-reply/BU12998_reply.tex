\documentclass[11pt]{article}
\usepackage{amsmath}
\usepackage{fullpage}
\usepackage{framed}
\usepackage{color}
\usepackage{graphicx}
\usepackage[backref,colorlinks,linkcolor=blue,citecolor=blue]{hyperref}
\usepackage{parskip}

\definecolor{shadecolor}{rgb}{0.757, 0.882, 0.925}

\begin{document}

Dr. Athanasios Chantis\hfill{\today}\\
Assistant Editor\\
Physical Review B

Dear Dr. Athanasios Chantis,\\

Please find here enclosed the revised version of the manuscript
BU12998/Anderson.

We thank the referees for their positive and useful comments and suggestions of
our manuscript and we believe that they have helped us to improve the quality of
our work.

In the following, we answer all the points raised by the referees and we also
list all the changes made in the paper. In the \verb=pdf= version we have marked
changes in red for easy reference. Below, our responses are enclosed in blue
boxes.

Additionally, in Fig. 7 we changed the curve labels in the key to italic letters
to maintain the same the notation used in the manuscript. This is purely a
cosmetic change that in no way affects the results depicted in the figure.

We are confident that all suggestions have been met, and hope that our
manuscript can now be published in Physical Review B.


\section{Report of the First Referee -- BU12998/Anderson}

The manuscript of Anderson and Mendoza is a comprehensive work devoted to the ab
initio calculation of the second harmonic generation from the silicon surfaces.
The incorporation of the multiple reflections of the fundamental and the second
harmonic fields is its distinctive feature. SHG susceptibility tensor is taken
as an input from the band structure calculations performed using the supercell
approach. In general, the work is well written, the results are sound and they
indeed show a substantial improvement over the calculations where multiple
reflections were not taken into account. However, we should not forget that the
theory is rather cumbersome, and some polishing of notations is needed.
Therefore, I recommend the publication with a minor revision.

Following points need to be addressed: 

\begin{enumerate}

\item
Clear statement about the position of this work within already published
results. In particular, I feel that authors contradict themselves when
emphasizing the novelty and necessity of incorporation of multiple reflections
and, at the same time, mentioning that ref. 18 (published already in 1962) ``was
the first to develop a model to describe the second-harmonic (SH) radiation that
considers the influence of multiple reflections in the reflected SH field''.
\begin{shaded}
We have refined that particular sentence (page 1, second paragraph) to more
accurately represent the work of Ref. 18, and added a sentence at the top of
page 2 to situate our work in this context.
\end{shaded}

\item
Eqs. (46), (56), are perhaps written in the shortest possible form. However, it
does not provide any physical insight. As an alternative consider a matrix form
of these (or similar) equations such as Eq. (3) of T. Andersen and W. H\"ubner,
PRB 65, 174409 (2002).
\begin{shaded}
We have implemented this excellent suggestion for the main expressions of
$r_{iF}$, and they are much easier to interpret. We also added matrix
representations for each surface symmetry ($\boldsymbol{\chi}^{(111)}$,
$\boldsymbol{\chi}^{(110)}$, and $\boldsymbol{\chi}^{(001)}$), which leads to a
much more visual and straightforward derivation of the results. See pages 7, 8,
and 9 for these changes.
\end{shaded}

\item
Another interesting aspect of this study that is worth mentioning is the total
internal reflection for the second harmonic light and corresponding
Goos-H\"anchen shift. Such possibility was discussed by Bloembergen and Pershan
(Ref. 18), recently the Goos-H\"anchen shift was predicted for SH signal (Optics
Express 21, 10878 (2013)) and observed experimentally (Sci. Rep. 6:19319
(2016)). However, no first principles calculations are available.
\begin{shaded}
We have added a note at the end of the second paragraph of page 1 concerning the
Goos-H\"anchen effect, with appropriate references.
\end{shaded}

\item
Finally, I have a conceptual question: If we compute the SHG susceptibility
tensor fully ab initio by taking all the geometry effects (i.e. presence of 3
layers) into account via the supercell approach should not it already
incorporate all the light scattering effects? After all, chi is computed for the
composite 3 layers system and not for a free standing central layer, cf.
statement in the introduction ``To mimic the semiinfinite system, we construct a
supercell consisting of a finite slab of material plus a vacuum region''. New J.
Phys. 14 (2012) 093044 is a demonstration of such an interplay between the
quantum-mechanical and optical approach to the second harmonic generation.
\begin{shaded}
Indeed, if such an approach can be followed for a semiconductor we agree that
all the light scattering effects would be incorporated. However, such an
approach is not what we used here as we have split the solution into two steps.
First, we calculate $\chi^{\mathrm{abc}}$ following quantum mechanical time
dependent perturbation theory; secondly, we use the $\chi^{\mathrm{abc}}$
components to calculate the radiated SH field and the SSHG yield. What the
referee suggests is a rather complicated problem, since it also includes the
self-consistent calculation of the electric field and the local field effects at
the same time that one obtains the linear and the non-linear response. This is a
problem of great interest and worth pursuing, but out of the scope of our
manuscript.

The article referred by the referee deals with a metallic system, and as claimed
by the authors in page 12, ``We do not pursue here a fully atomistic approach,
but rather solve the radial Kohn-Sham equation (using the renormalized Numerov
method) in the presence of the spherically symmetric ionic density.'' As a
consequence, their method does not produce a tensor for the nonlinear
susceptibility; rather, it produces what they call on page 9 ``The $\ell$-pole
frequency-dependent polarizability,'' $\alpha_{\ell  m}(\omega)$ (a scalar
quantity), which they calculate for the linear and the nonlinear
response of a metallic Na$^{-}_{2869}$ cluster, and no SHG yield is reported.
Based on this, we do not cite this article since it concerns a different
problem from the one treated in our manuscript.
\end{shaded}

\end{enumerate}


\section{Report of the Second Referee -- BU12998/Anderson}

The manuscript is well written and has the detailed theoretical analysis of
surface second harmonic generation (SSHG) yield from the surface of a
centrosymmetric material. The authors propose a three-layer model and have
derived the complete expressions to calculate the SSHG yield taking into account
the effects of multiple reflections inside the material from both the second
harmonic and fundamental fields.

They have used their new expressions (derived in detail) in the manuscript to
successfully explain the past experimental observations. According to the
authors, this work is an upgrade to the previously published article (reference
21) and now captures the effect of multiple reflections from the surfaces.
However, the authors do not present any case where the importance of this
improvement can be seen. It would be of help to bring out the importance of this
work by at least showing something new that this model is now able to predict.
This manuscript appears a good theoretical development to me but seems to lack
the impact.
\begin{shaded}
In this manuscript, we present a treatment that considers multiple reflections
in our theory of SSHG, we fully derive the expressions, and then we apply them
to two Si surfaces. We consider that these two surfaces showcase the features
that the developed theory is now able to explain and predict.

On the one hand, we consider the Si(111)(1$\times$1):H surface. We have
experimental spectra that were taken over a wide range of energies and in
absolute units. We show, through Figs. 3, 4, and 5, that including the effects
of multiple reflections improves the similarity between theory and experiment.
In agreement with the referee's suggestion, to further present how this is an
upgrade over our previous results, we add Tables II and III (see pages 10 and 11
with the corresponding text) that compare the yield ratio between the values of
the $E_{2}$ and $E_{1}$ peaks for $\mathcal{R}_{pP}$ and $\mathcal{R}_{pS}$. We
consider that these tables clearly show how the proportional height of the
theoretical peaks is closest to experiment when including the effects of
multiple reflections. This validates the physics behind our model for SSHG and
demonstrates what this model is now able to predict. As we are interested in
developing and implementing the most complete theory for SSHG, we think that
this is a step in the right direction.

On the other hand, we also present predictive calculations for the
Si(001)(2$\times$1) surface. The robust theoretical framework developed in the
manuscript can easily account for surfaces like these that lack any symmetry
considerations. We present ``heatmap'' plots that can guide the interested
experimenter, and demonstrate how the SH spectra of this surface is entirely
dominated by the topmost dimer layer. Although we do not have experimental data
to compare with, we consider these results to be quite interesting in their own
right, and should motivate their experimental verification.
\end{shaded}

Also the section numbers at the end of each reference should be removed.
\begin{shaded}
We have removed the section numbers from the end of each reference.
\end{shaded}

Overall I feel that the manuscript in its current form does not meet PRB's
criteria. The authors should revise the manuscript to clearly bring out the
significant impact of this theoretical work.

\end{document}
