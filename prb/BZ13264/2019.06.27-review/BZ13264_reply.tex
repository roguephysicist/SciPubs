\documentclass[aps,prb,10pt,endfloats]{revtex4-1}
\usepackage{amsmath}
\usepackage{framed}
\usepackage{color}
\usepackage[backref,colorlinks,linkcolor=blue,citecolor=blue,urlcolor=blue]{hyperref}
\usepackage{parskip}

\definecolor{shadecolor}{rgb}{0.757, 0.882, 0.925}

\begin{document}

\noindent Dr. Jason T. Haraldsen\hfill{\today}\\
Assistant Editor\\
Physical Review B\\

\noindent Dear Dr. Haraldsen,\\

Attached is the revised version of our paper entitled \emph{Plasmon dispersion
in graphite: A comparison of current ab initio methods}, manuscript
BZ13264/Anderson. Our response to the reviewer comments is included below.
Changes in the main manuscript text are marked in red in the PDF file for easy
reference.\\

We are confident that all suggestions have been met and that our manuscript is
now acceptable for publication in Physical Review B.\\

\noindent Best regards,\\

\noindent Dr. Sean M. Anderson\\
\href{mailto:sma@cio.mx}{sma@cio.mx}

%%%%%%%%%%%%%%%%%%%%%%%%%%%%%%%%%%%%%%%%%%%%%%%%%%%%%%%%%%%%%%%%%%%%%%%%%%%%%%%%
%%%%%%%%%%%%%%%%%%%%%%%%%%%%%%%%%%%%%%%%%%%%%%%%%%%%%%%%%%%%%%%%%%%%%%%%%%%%%%%%

\section{Second Report of the First Referee -- BZ13264/Anderson}

In the revised version of their manuscript, Anderson and colleagues have
addressed in detail my technical criticisms (I am the first referee) and the
recommendations of the second reviewer. Regarding the latter, I feel the changes
are adequate. In the shortened paper the motives are clearer and the text is
more readable and easy to follow. Regarding my own criticisms, it appears that
my misgivings about convergence were in fact due to use of incorrect data for
$q\rightarrow 0$ in the initial manuscript. I am happy to see that the new
datasets yield the correct quadratic dispersion. The authors have also provided
a convincing and demonstration of their convergence tests for RPA (being a
benchmark study, the authors may consider providing some of this analysis in the
expanded supplementary material).
\begin{shaded*}
Thank you for your kind remarks. Per your suggestion, we have added the
convergence report to the supplemental material.
\end{shaded*}

Regarding the broadening, I accept their point that neglecting coupling results
in intensities twice as large than without, this is clear. However, I still feel
their comment ``Methods that include these (BSE CP, ALDA CP, and RPA CP) have a
peak intensity between 0.5 and 1, which compares favorably with experiment''
should be better qualified or restated: the calculations using a smaller
broadening in the convergence tests yield peaks of about 1.2 at $q=0$
(unfortunately, there is no comparison provided for $q=0.21$). I don't mean to
badger the authors on this -- it only a small point -- but it remains unclear if
the favorable agreement with experiment is due to the somewhat arbitrary choice
of broadening or not.
\begin{shaded*}
We have modified the aforementioned phrase to be more generally representative
of the trend between methods, and made specific mention of the applied
broadening.
\end{shaded*}

Note some spelling errors have crept in: dielctric, calcations 
\begin{shaded*}
Our mistake -- we have corrected these errors.
\end{shaded*}

My main criticism of the work remains, however. It serves as a benchmark paper
whose conclusions may only be important for few systems (semi metals?); there is
little new physics. The authors acknowledge that it would be nice to provide
such a benchmark on a newer system of interest, but rightfully note that such a
study would introduce difficult convergence issues and that for experimental
reasons, graphite is a perfect material to tackle.

In summary, my opinion is that the paper may be suitable for publication in
Physical Review.
\begin{shaded*}
Thank you for your final assessment of the manuscript, and for noting that
graphite does indeed present a well established system for carrying out our
review of methods, including solving the full-excitonic Hamiltonian that
requires significant computational expense. We strove to create a well-rounded
and thorough manuscript with an in-depth literature/theory review, and a
systematic analysis of the available theoretical methods on an established
system. With your suggestions, we feel that we have improved considerably on our
original manuscript.
\end{shaded*}

\end{document}
