\documentclass{article}
\usepackage{amsmath}
\usepackage{framed}
\usepackage{color}
\usepackage{parskip}
\usepackage{fullpage}

\usepackage[backref,colorlinks,linkcolor=blue,citecolor=blue,urlcolor=blue]{hyperref}

\definecolor{shadecolor}{rgb}{0.757, 0.882, 0.925}

\begin{document}

\noindent Dr. Yan Li\hfill{\today}\\
Associate Editor\\
Physical Review B\\

\noindent Dear Dr. Li,

Attached is the revised version of our paper entitled \emph{Ab initio modeling
of optical functions after strain wave perturbation for defect detection},
manuscript BD13947/Anderson. We hope that we have resolved the very valid point
of the first reviewer, and that our manuscript is now acceptable for publication
in Physical Review B.

\noindent Best regards,

\noindent Dr. Sean M. Anderson\\
\href{mailto:sma@cio.mx}{sma@cio.mx}

%%%%%%%%%%%%%%%%%%%%%%%%%%%%%%%%%%%%%%%%%%%%%%%%%%%%%%%%%%%%%%%%%%%%%%%%%%%%%%%%
%%%%%%%%%%%%%%%%%%%%%%%%%%%%%%%%%%%%%%%%%%%%%%%%%%%%%%%%%%%%%%%%%%%%%%%%%%%%%%%%

\section{Second Report of the First Referee -- BD13947/Anderson}

In the revised version of the manuscript, the authors have addressed my most
important worry about their approach (I am the first referee) by making an
explicit connection to the time evolution of the signal in Fig. 4. The new
material about the electron density around the defect level is a welcome
addition to the manuscript and should help the reader visualize the system.
Besides the following minor point, I recommend that the manuscript is accepted
for publication in Phys. Rev. B.

Both the second referee and I raised the issue of the different scissors shifts
used for surface and bulk susceptibilities. Most importantly, the authors notes
in their reply that the exact value used has a negligible influence on the final
results. However, they have not modified the manuscript to explain better the
values used (except to clarify that 3.4eV is the experimental ``direct'' optical
gap, as I requested, corresponding to the E1 critical point in the JDOS), and
any reader with familiarity with DFT calculations of silicon would have a
reaction similar to ours. Having looked at the references cited by Referee 2, as
well as the authors' previous publications on the system, I have to say I cannot
remotely understand where they obtain the value of 0.98 eV for bulk Si from. To
quote:

``a scissors shift of 0.98 eV was used for the bulk calculation in order to
adjust the theoretical band gap to the experimental value of 3.4 eV for the
direct band gap.''

This implies the position of E1 in their DFT-LDA calculation was 2.4eV. This is
far lower than any calculation I have seen in the literature. Or to put things
differently, a bulk Si GW calculation yields shifts ranging from 0.7eV (Botti
PRB 10.1103/PhysRevB.69.155112) to 0.95eV (Adolph PRB 10.1103/PhysRevB.53.9797).
This yields spectra that overshoot the experimental spectra by a significant
amount: hence Schmidt (PRB 10.1103/PhysRevB.61.7604) uses a 0.5eV shift to match
with experiment, as the second referee notes. Li (Ref 59) even state
``Self-energy corrections to surface states are larger in magnitude by 0.2--0.3
eV than those to bulk states in the same energy range'', which would imply the
bulk correction is smaller, not larger, than the surface correction of 0.7eV.

Last, it is anyway odd/inconsistent to use a GW-computed value for the surface,
and an empirical value for the bulk.

I, therefore, urge the authors to rectify this matter, either in a resubmitted
version or a footnote added at the proof stage. It does NOT preclude acceptance
in Phys Rev. B, but in my opinion, it looks like an error that can easily be
corrected.\\

\begin{shaded*}
Thank you for your shrewd and accurate remarks concerning our values for the
scissors shift.

We are in complete agreement that there is no justifiable reason to use a
different value for the bulk. Therefore, we have recalculated every result
presented in the manuscript using the same 0.7\,eV value for both surface and
bulk susceptibilities. We have amended the phrase on page 3 (red text) to
reflect this change.

The trends of these new results (and our original conclusions), remain the same
as in the previous version of the manuscript; however, there are some minor
changes to Figs. 2--4 (in the new version of the manuscript) that reflect the
new numerical values.

We hope that with these corrections you can recommend acceptance in Phys.
Rev. B.
\end{shaded*}

\end{document}
