\documentclass[aps,prb,10pt,endfloats]{revtex4-1}
\usepackage{amsmath}
\usepackage{framed}
\usepackage{color}
\usepackage{parskip}
\usepackage{graphicx}

\usepackage[backref,colorlinks,linkcolor=blue,citecolor=blue,urlcolor=blue]{hyperref}

\definecolor{shadecolor}{rgb}{0.757, 0.882, 0.925}

\begin{document}

\noindent Dr. Yan Li\hfill{\today}\\
Associate Editor\\
Physical Review B\\

\noindent Dear Dr. Li,\\

Attached is the revised version of our paper entitled \emph{Ab initio modeling
of optical functions after strain wave perturbation for defect detection},
manuscript BD13947/Anderson. We thank the referees for their useful comments and
suggestions, as they have helped us to significantly improve the quality of our
work.\\

In the following, we answer all the points raised by the referees and list all
the changes made to the paper. We have marked changes in red for easy reference
in the revised PDF version of the manuscript. Our responses to the reviewer
comments are enclosed in blue boxes below.\\

We are confident that all suggestions have been met and consider that our
manuscript is now acceptable for publication in Physical Review B.\\

\noindent Best regards,\\

\noindent Dr. Sean M. Anderson\\
\href{mailto:sma@cio.mx}{sma@cio.mx}

%%%%%%%%%%%%%%%%%%%%%%%%%%%%%%%%%%%%%%%%%%%%%%%%%%%%%%%%%%%%%%%%%%%%%%%%%%%%%%%%
%%%%%%%%%%%%%%%%%%%%%%%%%%%%%%%%%%%%%%%%%%%%%%%%%%%%%%%%%%%%%%%%%%%%%%%%%%%%%%%%

\section{Report of the First Referee -- BD13947/Anderson}

Anderson, Mendoza, and Carriles present a first principles model of strain wave
propagation through a Si(111):H slab and study how the linear optical
reflectance changes as a result of a deep lying defect layer below the surface.
The idea is to simulate the results of coherent phonon spectroscopy (CAP). The
strain wave is modeled analytically and the optical response (change in
reflectance) is computed in the independent particle picture. As both strain and
light penetrate deeply into semiconductor slabs, and the the strain profile
changes at each timestep, an ab initio approach is an ambitious undertaking.
Here the authors use a very thick slab of 150 A, which forces them to use a 1x1
cell and thus consider a simple defect represented by a displacement of one
atomic layer. The authors acknowledge this is an unrealistic situation to
produce experimentally, and while localized/point defects are beyond treatment,
their approach may work for buried interfaces, dislocations, thin films, and so
on. The paper is well written, the figures are clear, the analysis of the
obtained spectra is interesting, and the approach, to my eyes, looks reasonable.
\begin{shaded*}
We thank the reviewer for the assessment and comments about our work. We have
addressed the issues raised by the reviewer in what follows below.
\end{shaded*}

I recommend publication after the authors have considering the following:

\begin{enumerate}

\item My main worry about the approach is that the final signal (fig 2 b) 
clearly reveals a new peak related to the defect, but does not seem sensitive
enough to allow an experimentally measured defect position to be inferred (the
differences between the various profiles is very small). The interesting change
in reflectance with respect to the defect depth (Fig 4a) is I suppose integrated
into the main peak in Fig 2b and thus the information is hidden. Is this
correct?
\begin{shaded*}
Actually, the results presented in Fig. 4 are not integrated into any of the
peaks featured in Fig. 2; rather, Fig. 4 represents the time-evolution of
$\Delta R$ as the strain wave passes through the material. Each of the three
panels represents a different spectral energy range, with panel (a)
corresponding to the energy range of the defect peaks in the dashed box of Fig.
2.

As the strain wave maximum passes through the defect layer, there is an abrupt
change in sign in the defect peak of the $\Delta R$ spectrum, that is not
present in the other energy ranges. In order to elucidate the defect layer
depth, the measurement would only have to be taken (for this particular
material) over the 2.5--3.0\,eV energy range, and when the $\Delta R$ signal has
an abrupt sign change, we know we have passed the defect layer.

To make this clearer, we have added the appropriate time scale to the top of
Fig. 4, which relates directly to the depth of the strain wave via the
longitudinal velocity of sound in Si, which is 8433\,m\,s$^{-1}$ which is given
in Table 1 of the manuscript. We have also added a brief explanation of this
trend at the end of the third paragraph of page 5.
\end{shaded*}

\item The reason for the different scissors shifts for surface and bulk is not
clear to me. I suppose that for a slab as thick as the one used here, the size
quantization is minimal and the bulk states are well represented. Also: 3.4eV
refers to the direct gap, which should be noted.
\begin{shaded*}
There are two susceptibilities required to calculate the reflectance:
$\chi^{aa}_{sc}(\omega)$, which is the linear susceptibility for the supercell
system, and $\boldsymbol{\chi}^{~}_{b}(\omega)$ which is the linear
susceptibility of the isotropic bulk crystal. The former, is a surface response,
while the latter is a bulk response; therefore, the optical band gap is not
necessarily the same. For this reason, different scissors values are used.

In this case, we used a $G_{0}W_{0}$ value (0.7\,eV) for the surface
calculation; for the bulk calculation, we used a scissors shift of 0.98\,eV to
obtain the experimental value of the optical band gap. However, we have verified
our results by using the same $G_{0}W_{0}$ shift of 0.7\,eV for both surface and
bulk responses. The difference between the results obtained from using a bulk
scissors shift of 0.7 and 0.98\,eV was negligible.
\end{shaded*}

\item Can the authors explain the origin of the 2.8eV peak in terms of some
localized defect state, rather than some general bond length argument? It may be
useful to understand what kind of perturbation the approach is sensitive to.
\begin{shaded*}
This is a very interesting point for which we have added Fig. 5 and the
associated explanation on page 6 of the manuscript. We think that this new
material adds more value to the work, and strengthens the overall physical
message being presented.
\end{shaded*}

\item Note some typos: page 6 ``present present'' and ``were we plot''
\begin{shaded*}
We have corrected the aforementioned typos, and have re-checked the grammar and
spelling of the manuscript.
\end{shaded*}

\end{enumerate}

%%%%%%%%%%%%%%%%%%%%%%%%%%%%%%%%%%%%%%%%%%%%%%%%%%%%%%%%%%%%%%%%%%%%%%%%%%%%%%%%
%%%%%%%%%%%%%%%%%%%%%%%%%%%%%%%%%%%%%%%%%%%%%%%%%%%%%%%%%%%%%%%%%%%%%%%%%%%%%%%%

\section{Report of the Second Referee -- BD13947/Anderson}

The authors present an original and well-written theoretical study on the
influence of strain on surface optical properties. The work extends early
research by Hingerl and co-workers on the influence of static strain [see, e.g.,
App. Surf. Sci. 175/176 (2001) 769-776] (which probably should be cited in this
context) to the dynamic range and points out an interesting application, the
detection of buried defect layers.
\begin{shaded*}
We thank the reviewer for the assessment and insightful comments. We have added
the suggested reference in the first paragraph of the introduction.
\end{shaded*}

The methodology is sound and the calculations are well done. There is one point
I do not understand, however. The authors write (page 3) ``A scissors shift of
0.7 eV was used for the surface susceptibility, obtained from G0W0 calculations
[58]; a scissors shift of 0.98 eV was used for the bulk calculation'' A scissors
shift 0.98 eV is in fact very large for DFT calculations on silicon. In an
previous calculation on the optical response of Si(111):H [Phys. Rev. B 61, 7604
(2000)] a value of 0.5 eV was used. In Ref. [58] cited by the authors a
difference of 0.68 eV was found for bulk DFT and G0W0 calculation. The authors
should explain (i) how they arrived at the values of 0.7 and 0.98 eV and (ii)
how they apply these two different values to the calculation of the optical
response. Do they weight the shift according to the surface-localization of the
respective state?
\begin{shaded*}
We have explained this point in our reply to the first referee; please see point
2 above.
\end{shaded*}

Once this clarified, I suppose the paper will be suitable for publication.

\end{document}
